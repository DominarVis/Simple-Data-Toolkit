
\documentclass[11pt]{article}
\usepackage[utf8]{inputenc}
%\geometry{a4paper}
\usepackage{graphicx}
\usepackage{booktabs} % for much better looking tables
\usepackage{array} % for better arrays (eg matrices) in maths
\usepackage{paralist} % very flexible & customisable lists (eg. enumerate/itemize, etc.)
\usepackage{verbatim} % adds environment for commenting out blocks of text & for better verbatim
\usepackage{subfig} % make it possible to include more than one captioned figure/table in a single float
% These packages are all incorporated in the memoir class to one degree or another...

%%% HEADERS & FOOTERS
\usepackage{fancyhdr} % This should be set AFTER setting up the page geometry
\pagestyle{fancy} % options: empty , plain , fancy
\renewcommand{\headrulewidth}{0pt} % customise the layout...
\lhead{}\chead{}\rhead{}
\lfoot{}\cfoot{\thepage}\rfoot{}

%%% SECTION TITLE APPEARANCE
\usepackage{sectsty}
\allsectionsfont{\sffamily\mdseries\upshape} % (See the fntguide.pdf for font help)
% (This matches ConTeXt defaults)

%%% ToC (table of contents) APPEARANCE
\usepackage[titles,subfigure]{tocloft} % Alter the style of the Table of Contents
\renewcommand{\cftsecfont}{\rmfamily\mdseries\upshape}

\usepackage{makecell}

%%% END Article customizations

%%% The "real" document content comes below...

\title{Simple Log Grabber}
\author{Franklin E. Powers, Jr.}
%\date{} % Activate to display a given date or no date (if empty),
         % otherwise the current date is printed 

\begin{document}
\maketitle

\section{Introduction}

\paragraph{Simple Log Grabber was created to allow for simple manipulation of logs in different formats.  Additionally, it can remove entries.}
\paragraph{At the moment, there are only two ways to use it:}
\begin{enumerate}
    \item As a library, which you can integrate into your own programs
    \item As a command-line utility
\end{enumerate}
\paragraph{Starting the tool depends on platform and version as shown in the table below:}
\begin{center}
\begin{tabular}{ | c | c | }
\hline
C++ & slg \\
\hline
C\# & slg \\
\hline
Java & java -jar slg.jar \\
\hline
JScript Windows Shell & cscript slg-wsh.js \\
\hline
Python & python slg.py \\
\hline
\end{tabular}
\end{center}

\section{Supported Input Formats}
\begin{center}
\begin{tabular}{ | c | c | }
\hline
\makecell{ Default } & \makecell{ stdin/Standard Input is read for log entries }  \\
\hline
\makecell{ Executable } & \makecell{ Executes executable and reads from stdout/Standard Out }  \\
\hline
\end{tabular}
\end{center}

\section{Supported Output Formats}
\begin{center}
\begin{tabular}{ | c | c | }
\hline
\makecell{ Default } & \makecell{ Outputs log entries to timestamped files }  \\
\hline
\makecell{ EVENT } & \makecell{ Outputs log entries to Event Viewer }  \\
\hline
\end{tabular}
\end{center}

\section{Filters}
The filtering commands will be listed in order of precedence.
\\
\begin{center}
\begin{tabular}{ | c | c | }
\hline
\makecell{ INCLUDE } & \makecell{ This specifies a filter by which entries are included. } \\
\hline
\makecell{ EXCLUDE } & \makecell{ This specifies a filter by which entries are excluded. } \\
\hline
\end{tabular}
\end{center}

Filtering comes in the following forms:
\\
\begin{center}
\begin{tabular}{ | c | c | c | }
\hline
    \makecell{ Counting \\ List }
    & \makecell{ Starts with a hash/pound \\ (\#) sign and has a comma (,) \\ separated list. }
    & \makecell{ Indicates that the filter \\ is based on the index/count of \\ the row/column/etc and should \\ compare against the given list. } \\
\hline
    \makecell { Counting \\ Range }
     & \makecell{ Starts with a hash/pound \\ (\#) sign and has a minus sign \\ (-) between a lower and \\ upper limit. }
     & \makecell{ Indicates that the filter \\ is based on the index/count of \\ the row/column/etc and should \\ compare against the given lower \\ and upper limit (inclusive). } \\
\hline
    \makecell { Counting \\ Single }
    & \makecell{ Starts with a hash/pound \\ (\#) sign and is neither a list \\ or a range }
    & \makecell{ Indicates that the filter \\ is based on the index/count of \\ the row/column/etc and should \\ compare against the specified \\ value. } \\
\hline
    \makecell { Regular \\ Expression }
    & \makecell{ Starts and ends with a \\ forward slash (/) }
    & \makecell{ Indicates the filter is to \\ run the given regular \\ expression. } \\
\hline
    \makecell { String \\ Equals }
    & \makecell { Starts with an equals sign (=) }
    & \makecell{ Indicates the filter should \\ compare to the specified string \\ and if it matches then the filter \\ applies. } \\
\hline
    \makecell { String \\ With/Contains }
    & \makecell { Starts with no specific \\ character }
    & \makecell{ Indicates the filter should \\ compare to the specified string \\ and if it is found anywhere, then \\ the filter applies. } \\
\hline
\end{tabular}
\end{center}

\section{Examples}
\paragraph{Read stdout from old script file and send logs to timestamped file}
\begin{verbatim}
    slg script.js
\end{verbatim}
\paragraph{Read stdout from old script file, send logs to timestamped file, but only read entries that contain ERROR}
\begin{verbatim}
    slg script.js INCLUDE ERROR
\end{verbatim}
\paragraph{Redirect output from command to slg and send to Event Viewer}
\begin{verbatim}
    command | slg EVENT
\end{verbatim}
\paragraph{Read log file, send output to Event Viewer, but only read entries that contain WARNING}
\begin{verbatim}
    slg EVENT INCLUDE ERROR < file.txt
\end{verbatim}


\end{document}
