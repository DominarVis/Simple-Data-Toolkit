
\documentclass[11pt]{article}
\usepackage[utf8]{inputenc}
%\geometry{a4paper}
\usepackage{graphicx}
\usepackage{booktabs} % for much better looking tables
\usepackage{array} % for better arrays (eg matrices) in maths
\usepackage{paralist} % very flexible & customisable lists (eg. enumerate/itemize, etc.)
\usepackage{verbatim} % adds environment for commenting out blocks of text & for better verbatim
\usepackage{subfig} % make it possible to include more than one captioned figure/table in a single float
% These packages are all incorporated in the memoir class to one degree or another...

%%% HEADERS & FOOTERS
\usepackage{fancyhdr} % This should be set AFTER setting up the page geometry
\pagestyle{fancy} % options: empty , plain , fancy
\renewcommand{\headrulewidth}{0pt} % customise the layout...
\lhead{}\chead{}\rhead{}
\lfoot{}\cfoot{\thepage}\rfoot{}

%%% SECTION TITLE APPEARANCE
\usepackage{sectsty}
\allsectionsfont{\sffamily\mdseries\upshape} % (See the fntguide.pdf for font help)
% (This matches ConTeXt defaults)

%%% ToC (table of contents) APPEARANCE
\usepackage[titles,subfigure]{tocloft} % Alter the style of the Table of Contents
\renewcommand{\cftsecfont}{\rmfamily\mdseries\upshape}

\usepackage{makecell}

%%% END Article customizations

%%% The "real" document content comes below...

\title{Simple Calendar}
\author{Franklin E. Powers, Jr.}
%\date{} % Activate to display a given date or no date (if empty),
         % otherwise the current date is printed 

\begin{document}
\maketitle

\section{Introduction}

\paragraph{Simple Calendar was created to allow for simple manipulation ICS and other calendar related formats.}
\paragraph{At the moment, there are only two ways to use it:}
\begin{enumerate}
    \item As a library, which you can integrate into your own programs
    \item As a command-line utility
\end{enumerate}
\paragraph{This document will cover how to use it as a command-line utility.}
\paragraph{Starting the tool depends on platform and version as shown in the table below:}
\begin{center}
\begin{tabular}{ | c | c | }
\hline
C++ & calendar \\
\hline
C\# & calendar \\
\hline
Java & java -jar calendar.jar \\
\hline
JScript Windows Shell & cscript calendar-wsh.js \\
\hline
Python & python calendar.py \\
\hline
\end{tabular}
\end{center}

\section{Creating an Event}
\paragraph{First datetime specified is the start datetime.  Second datetime specified is the end datetime.  First non-filename string specified is the summary of the event.}
\paragraph{When specifying a filename, if not datetime and summary are specified, then it is considered an input file and it's contents are displayed on screen.}
\paragraph{If a filename, datetime, and summary are specified, then the filename is considered an output file and the date is written to the file.}
\paragraph{If two filenames are specified, then it considers the first filename to be an input file and the second to be an output file.}

\section{Examples}
\paragraph{Displays contents of an ics file}
\begin{verbatim}
    calendar event.ics
\end{verbatim}
\paragraph{Create new event and store in an ics file}
\begin{verbatim}
    calendar "2019-07-01 10:00:00 AM" "2019-07-01 11:00:00 AM" "Visiting A Friend's House" event.ics
\end{verbatim}

\end{document}
