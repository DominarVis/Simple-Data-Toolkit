
\documentclass[11pt]{article}
\usepackage[utf8]{inputenc}
%\geometry{a4paper}
\usepackage{graphicx}
\usepackage{booktabs} % for much better looking tables
\usepackage{array} % for better arrays (eg matrices) in maths
\usepackage{paralist} % very flexible & customisable lists (eg. enumerate/itemize, etc.)
\usepackage{verbatim} % adds environment for commenting out blocks of text & for better verbatim
\usepackage{subfig} % make it possible to include more than one captioned figure/table in a single float
% These packages are all incorporated in the memoir class to one degree or another...

%%% HEADERS & FOOTERS
\usepackage{fancyhdr} % This should be set AFTER setting up the page geometry
\pagestyle{fancy} % options: empty , plain , fancy
\renewcommand{\headrulewidth}{0pt} % customise the layout...
\lhead{}\chead{}\rhead{}
\lfoot{}\cfoot{\thepage}\rfoot{}

%%% SECTION TITLE APPEARANCE
\usepackage{sectsty}
\allsectionsfont{\sffamily\mdseries\upshape} % (See the fntguide.pdf for font help)
% (This matches ConTeXt defaults)

%%% ToC (table of contents) APPEARANCE
\usepackage[titles,subfigure]{tocloft} % Alter the style of the Table of Contents
\renewcommand{\cftsecfont}{\rmfamily\mdseries\upshape}

\usepackage{makecell}

%%% END Article customizations

%%% The "real" document content comes below...

\title{Simple Table Converter}
\author{Franklin E. Powers, Jr.}
%\date{} % Activate to display a given date or no date (if empty),
         % otherwise the current date is printed 

\begin{document}
\maketitle

\section{Introduction}

\paragraph{Simple Table Converter was created to allow for simple manipulation of tables in different formats.  Additionally, it can remove different rows and columns.}
\paragraph{At the moment, there are only two ways to use it:}
\begin{enumerate}
    \item As a library, which you can integrate into your own programs
    \item As a command-line utility
\end{enumerate}
\paragraph{This document will cover how to use it as a command-line utility.}
\paragraph{Starting the tool depends on platform and version as shown in the table below:}

\begin{center}
\begin{tabular}{ | c | c | }
\hline
C++ & stc \\
\hline
C\# & stc \\
\hline
Java & java -jar stc.jar \\
\hline
JScript Windows Shell & cscript stc-wsh.js \\
\hline
Python & python stc.py \\
\hline
\end{tabular}
\end{center}

\section{Converting Table Formats}
\paragraph{STC always converts from one source table to one target table.  Either can be specified on the command-line.  When a source table isn’t specified, stdin (standard input/comsole input) is used.  When a target table isn’t specified, stdout (standard output/console output) is used.  By default, the table format is specified by the filename extension.  If no format can be determined, then the default is CSV.}

\subsection{Supported File Formats}
\begin{center}
\begin{tabular}{ | c | c | }
\hline
CSV & Comma Separated Values file \\
\hline
PSV & Pipe Separated Values file \\
\hline
TSV & Tab Separated Values file \\
\hline
\end{tabular}
\end{center}

\section{Filters}
The filtering commands will be listed in order of precedence.
\\
\begin{center}
\begin{tabular}{ | c | c | }
\hline
INCLUDEROWS & \makecell{ This specifies a filter by which rows are included. \\ The filter runs on the name (or first column) of the row. } \\
\hline
EXCLUDEROWS & \makecell{ This specifies a filter by which rows are excluded. \\ The filter runs on the name (or first column) of the row .} \\
\hline
INCLUDECOLUMNS & \makecell{ This specifies a filter by which columns are included. \\ The filter runs on the name (or first row) of the column. } \\
\hline
EXCLUDECOLUMNS & \makecell{ This specifies a filter by which columns are excluded. \\ The filter runs on the name (or first row) of the column. } \\
\hline
\end{tabular}
\end{center}

Filtering comes in the following forms:
\\
\begin{center}
\begin{tabular}{ | c | c | c | }
\hline
    \makecell{ Counting \\ List }
    & \makecell{ Starts with a hash/pound \\ (\#) sign and has a comma (,) \\ separated list. }
    & \makecell{ Indicates that the filter \\ is based on the index/count of \\ the row/column/etc and should \\ compare against the given list. } \\
\hline
    \makecell { Counting \\ Range }
     & \makecell{ Starts with a hash/pound \\ (\#) sign and has a minus sign \\ (-) between a lower and \\ upper limit. }
     & \makecell{ Indicates that the filter \\ is based on the index/count of \\ the row/column/etc and should \\ compare against the given lower \\ and upper limit (inclusive). } \\
\hline
    \makecell { Counting \\ Single }
    & \makecell{ Starts with a hash/pound \\ (\#) sign and is neither a list \\ or a range }
    & \makecell{ Indicates that the filter \\ is based on the index/count of \\ the row/column/etc and should \\ compare against the specified \\ value. } \\
\hline
    \makecell { Regular \\ Expression }
    & \makecell{ Starts and ends with a \\ forward slash (/) }
    & \makecell{ Indicates the filter is to \\ run the given regular \\ expression. } \\
\hline
    \makecell { String \\ Equals }
    & \makecell { Starts with an equals sign (=) }
    & \makecell{ Indicates the filter should \\ compare to the specified string \\ and if it matches then the filter \\ applies. } \\
\hline
    \makecell { String \\ With/Contains }
    & \makecell { Starts with no specific \\ character }
    & \makecell{ Indicates the filter should \\ compare to the specified string \\ and if it is found anywhere, then \\ the filter applies. } \\
\hline
\end{tabular}
\end{center}

Sorting can be done with:
\\
\begin{center}
\begin{tabular}{ | c | c | }
\hline
ORDERBY & \makecell{ Orders the rows by the columns specified in the comma separated list. } \\
\hline
\end{tabular}
\end{center}

Automated testing can be executed with the following parameters:
\\
\begin{center}
\begin{tabular}{ | c | c | }
\hline
RUNTESTS & \makecell{ Executes the unit tests and determines if the resulting executable \\ works as expected.  Only failed tests are outputted by default. } \\
\hline
VERBOSE & \makecell{ Outputs additional details about the test results. } \\
\hline
RECORDPASS & \makecell{ Outputs the results for tests that pass. } \\
\hline
\end{tabular}
\end{center}

\section{Examples}
\paragraph{Convert the CSV file in.csv to a TSV file out.tsv.}
\begin{verbatim}
    stc in.csv out.tsv 
\end{verbatim}
\paragraph{Remove the 3rd column from the CSV file in.csv and store in out.csv.}
\begin{verbatim}
    stc in.csv out.csv excludecolumns #3
\end{verbatim}
\paragraph{Keep only the columns that have A in the header and store in out.csv.}
\begin{verbatim}
    stc in.csv out.csv includecolumns A
\end{verbatim}
\paragraph{Remove the 3rd, 4th, and 5th columns from the CSV file in.csv and store in out.csv.}
\begin{verbatim}
    stc in.csv out.csv excludecolumns #3,4,5
    stc in.csv out.csv excludecolumns #3-5
\end{verbatim}


\end{document}
